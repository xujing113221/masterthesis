\chapter*{Abstract}
\label{chap:Abstract}
\addcontentsline{toc}{chapter}{Abstract}
\setcounter{page}{1}
\pagenumbering{roman}
%******************************************************************************
In this thesis we mainly discuss  the solutions for visual relation detection.To find the solutions, we need to not only develop models that detect the objects in an image, but also predict the interactions between the detected objects. Therefore, a comprehensive scene understanding of an image would be considered in aspect of connecting computer vision and natural language  helpful. Even though the work in field of deep learning has been quite accomplished, but it remaining to be a complex and challenging task to extract related scene information. 

 Inspired by the contemporary achievement of transformer in computer vision, we propose Retina Net, which was based transformer structure designed . In order to meet different requirements in visual relation detection, we designed specific object query with physical meanings. The specific object query is capable of extracting better object features in an image with the help of our attention loss.  By taking into account global context, we also model relation to object interactions through our designed relation decoder. We attempt a lot to find the mechanism of attention in the models and to understand which details should be paid more attention to when it comes to relation predicts.
 
 To our best effort, by using transformer structure our study for the first time try to complete all the models of relation detection . It is a new approach to solve the problems. The results of our experiments indicate that our models solve the tasks in visual relation detection properly.

%------------------------------------------------------------------------
%-------------------------------------------------------------------------

%******************************************************************************
\chapter*{Kurzfassung}

%------------------------------------------------------------------------
%-------------------------------------------------------------------------

%******************************************************************************
\label{chap:Kurzfassung}
\addcontentsline{toc}{chapter}{Kurzfassung}

In dieser Arbeit 