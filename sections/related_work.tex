\chapter{Related Work}
\label{chap:relatedwork}
During the last decades, a large number of studies have focused on computer vision-based behavior analysis. Most existing approaches fall into two categories: object-centric tracking based and non object-centric statistical approaches. 
In general, the models for activity recognition can be roughly divided into generative models such as HMM, which focuses on inferring the joint probability distribution, and discriminative models such as Gaussian Process models, which concentrates on modeling the conditional probability distribution.
In this chapter, we first discuss about the existing approaches for activities recognition. Then, we compare the advantages and disadvantage between generative models and discriminative models for the application of activity recognition. The well-known existing models and their limitation are overlooked. 
%At the end, we focus on the learning model because they are the center of our dissertation
