\chapter{Related Work}
\label{chap:relatedwork}
During the last decades, a large number of studies have focused on computer vision-based behavior analysis. Most existing approaches fall into two categories: object-centric tracking based and non object-centric statistical approaches. 
In general, the models for activity recognition can be roughly divided into generative models such as HMM, which focuses on inferring the joint probability distribution, and discriminative models such as Gaussian Process models, which concentrates on modeling the conditional probability distribution.
In this chapter, we first discuss about the existing approaches for activities recognition. Then, we compare the advantages and disadvantage between generative models and discriminative models for the application of activity recognition. The well-known existing models and their limitation are overlooked. 
%At the end, we focus on the learning model because they are the center of our dissertation


%******************************************************************************
\section{Tracking-based Approaches}
\label{related:tracking}
Tracking-based approaches for activities recognition have been long focused and proved to be effective due to its advantages especially in clear representation of trajectories of moving objects of interest. 
In the first, objects of interest are detected, tracked, and classified  into different object categories. Then, object tracks are used to model activities~\cite{kim2011gaussian, wang2011trajectory, Oliver_tracking_based, wang2014abnormal, aggarwal2011human, tiger2014towards}. 
For example, Tiger et al.~\cite{tiger2014towards} classified trajectories into different activities based on the moving directions, speeds, positions, sizes and silhouettes of objects of interest along the tracks. 
Wang et al.~\cite{wang2011trajectory} semantically analyzed the trajectories and clustered them into activities. 
Oliver et al.~\cite{Oliver_tracking_based} used a coupled HMM to model the interaction between two tracks. 
Kim et al.~\cite{kim2011gaussian} modeled a trajectory as a continuous dense flow field from a sparse set of vector sequences using Gaussian Process Regression. Possible paths are predicted and anomalous events from online trajectories detected, if the observed trajectories deviate from the prediction more than a threshold.
By means of tracking, the activity of one object can be separated from other co-occurring activities and the spatial state of visual objects over time is clearly represented. This allows behaviors such as typical flows of traffic to be cleanly modeled and clustered. The readily defined events in terms of individual trajectories such as move, stop, accelerating~\cite{kim2011gaussian} will be recognized and the abnormal activities such as U-turns or counter flow~\cite{berclaz2008multi} will be detected.

However, these tracking-based approaches depend crucially on the reliability of detection, tracking and/or recognition in both training and test data. Missing detection and tracking broken in a few frames may lead completely wrong results in the future tracks. On the other hand, most of these approaches are supervised. They model complicated activities and interactions \cite{ghanem2004representation}, such as ''vehicles stop and wait for pedestrian'', using the primitive events such as ''stop, move''. However, these primitive events must be first learned from labeled training examples. In crowded and complicated scenes, such as shopping mall, busy traffic junction, a train station, there are a large number of agents of interest, and the resolution of video are normal low. Detection and tracking cannot promise a well enough performance in such  complex scenes with crowded motions. 
It's hard to separate individual trajectory from the others, which indicate different activities, and manually labeling are time consuming. 
%In brief, the major limitation of tracking based approaches in general is the difficulty in modeling behaviors characterized by coordinated activity of multiple objects, which may be the defining characteristic of an interesting behavior in video.

%To improve robustness of activity models to tracking failures, non-parametric representations  of track statistics have also been exploited (Basharat  et  al.2008;Saleemi  et  al.2009). Nevertheless, a major limitation of tracking based approaches in general is the difficulty in modeling behaviors characterized by coordinated activity of multiple objects, which may be the defining characteristic of an interesting behavior in video.

To tackle the problems of missing detection and broken tracks,  many studies have focused on directly processing on image data \cite{zelnik2001event, zhong2004detecting, boiman2007detecting}. For example,  Zelnik-Manor and Irani \cite{zelnik2001event} and Zhong et al.~\cite{zhong2004detecting} directly use global motion feature vectors to describe video clips and clustered them into typical event categories. In these approaches, each video clip is treated as an integral entity and the whole clip is noted as normal or abnormal. They are only applicable in simple datasets where there is only one kind of activity in a video clip. It is hard for them to model multi-agent interactions because they cannot separate a particular activity from the other activities simultaneously existing in the same clip.

%******************************************************************************
\section{Statistical Approaches}
\label{related:statistical approaches}

To model more complex multi-agent activities and interactions, many studies have focused on developing more sophisticated statistical models using directly low-level visual features \cite{ zhong2004detecting, duong2005activity, benezeth2009abnormal, kuettel2010s, hospedales2011identifying, saleemi2010scene, wang2009unsupervised, hospedales2012video}. 
Typical approaches include Gaussian mixture models (GMM) \cite{saleemi2010scene}, Dynamic Bayesian Networks (DBNs) \cite{swears2014complex, vo2014stochastic} such as Hidden Markov Models (HMM) \cite{banerjee2014pose}, or probabilistic topic models (PTMs) \cite{kinoshita2014traffic} such as Latent Dirichlet Allocation (LDA) \cite{blei2003latent}, Hierarchical Dirichlet Process (HDP) \cite{teh2006hdp} or extensions \cite{wang2009unsupervised, hospedales2009markov, kuettel2010s}. 
DBNs have been widely applied and proved to be effective and robust models to modeling temporal dynamics of activity patterns beyond tracking \cite{xiang2006beyond, duong2005activity, banerjee2014pose}. However, learning an appropriate DBN model remains a difficult problem because of the predefined number of hidden states, the model topology, and the state connectivity, the so called DBN model complexity. Correctly estimating model complexity requires sufficiently large training sample size. Furthermore, it is a parametric model, the prior knowledge is necessary in advance. %Although the model can be well trained with enough training data and prior knowledge, it is a challenging to apply in surveillance video data due to its poor flexibility. The state
Besides, training an explicit DBN model of multiple object are exponentially costly in computation and time. Particularly, in a complicated and crowded scene these challenges are more oblivious because of large number of agents, not well defined activity states and dynamics.
To overcome these problems in DBNs, many studies have resorted to PTMs from word-document analysis.


Topic models have received increasing attention to analyze activity in surveillance video and their effectiveness and robustness have been proved~\cite{hospedales2009markov, hospedales2011identifying, kuettel2010s, li2008global, wang2009unsupervised}. 
In language processing domain, a document is represented as a bag of words by a unique mixture of intermediate topics. A topic defines a distribution over words in the codebook. Applied to behavior analysis in surveillance videos, a video clip is represented as a mixture of activities. Each activity is uniquely defined as distribution over visual words. Based on original LDA \cite{blei2003latent} and HDP \cite{teh2006hdp} models many extensions are devised for application in surveillance videos. 
For example, Wang et al.~\cite{ wang2009unsupervised} extended HDP to Dual-HDP, which allows clustering visual words into activities and video clips into traffic states at the same time. However, they are offline models and batch procedures. The temporal dependencies among behavior are neglected.
Kuettel et al.~\cite{kuettel2010s} proposed a hybrid model of Dpendent Dirichlet Process Hidden Markov Model (DDP-HMM). They learned an arbitrary number of Hidden Markov Models with an arbitrary number of states each using Dependent Dirichlet Processes. Temporal dependencies between activities and states could be well explained in this model. However, the inference of this model is prohibitively costly due to the infinite mixture of infinite HMMs and their approach still work as an offline batch process.
Hospedales et al.~\cite{hospedales2009markov, hospedales2011identifying} used LDA to models atom activities and enabled their model to work online. However, LDA is less powerful than HDP because it requires user to specify the number of clusters. It is hard to give a optimal number of possible activities that may occur in a video from a crowded scene. 
Besides, either LDA or HDP models need to perform iterative sweeps of the Gibbs sampler to estimate the joint distribution of input sample, which is time consuming and could be a limitation for online application. 

The PTMs have the advantage of fully automatic and unsupervised operation, and effectiveness and robustness in exploiting activities and states. However, these approaches have many crucial limitations. Firstly, they are often computationally expensive for inference. Secondly, as generative models, their accuracy is lower than discriminative models.


%******************************************************************************
\section{Supervised Classification Methods}
\label{related:classifier}
Supervised learning methods have also often been applied to human activity recognition and behavior analysis~\cite{kim2011gaussian, wang2008gaussian, deng2014cross}. They enjoy high accuracy and robustness in classification given sufficient and unbiased labeled training data. Different from unsupervised methods, they can deal with non-separable data and classify behaviors. Typical supervised classifier includes GP classifier~\cite{althloothi2014human}, SVM~\cite{deng2014cross} and Random Forest (RF)~\cite{baumann2013action}.

However, they require supervision for training and perform poorly if the target class has few examples. Moreover, the manual effort required to label training data may be prohibitively costly. On the other hand, classification methods are feature-based (descriptor) approaches. They have high requirement in the applicability and the preciseness of features to ensure their performance. The most widely used features include HOG feature, optic flow based features, scale-invariant feature transform (SIFT), etc. To describe an activity explicitly, the feature vectors may be high dimensional.
Therefore, many classification methods are applied to recognize simple human actions or activities.
To use classification methods to recognized complex interactions, many good descriptors have been designed over the years. For example, Tang et al.~\cite{tang2013combining} proposed a novel method to combine features for complex event recognition.
However, these features are extracted in the videos with clear background, low noise, good resolution and small number of agents such as KTH, Multi-media Event Detection (MED) dataset, etc. 
In practice, a surveillance video sequence over a complicated scene of crowded motions monitors a large scene and provides low resolution. Numerous agents with different motions present simultaneously. It is not easy to separate individual activity and agent to extract their motion features. The low-level motion features are more reliable than others in such cases.

Without detection and tracking, our approach automatically model activities and interactions using generative models, and recognizes high-level events in real-time using supervised classifier in complicated scenes. Different from the other approaches, we do not need a training dataset with manual label beforehand or prior knowledge. 
To improve the performance, the temporary dependencies are taken into account in our GP classifier.
To our best knowledge, our study is the first attempt to combine non-parametric generative HDP model and discriminant GP model for unsupervised online video event classification in a complicated and crowded traffic scene. 
