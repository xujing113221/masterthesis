\chapter{Conclusions and Future Work}
\label{chap:conclusion}
In the present thesis a novel unsupervised learning framework has been proposed to model the activities and interactions, to online recognize global interactions and to identify abnormal events in crowded and complicated traffic scenes. Through combining the advantages of both generative models and discriminative ones, the formulated approach provides an effective solution to the problems of high-level video events recognition and abnormal events detection in real time. 

(1) Owing to its computation efficiency as well as comparative reliability in the far-field surveillance data, the quantized optical flow is adopted in this thesis as the input vocabulary for the HDP models. However, the activities and interactions learned by HDP models are represented in the form of probabilistic distribution over low-level features, which is definitely inappropriate to be directly employed by the GP models. Therefore, a more effective method is proposed to represent activities and interactions as the input features of GP models. The proposed method uses the low-level features and middle-level ones to represent the middle-level events and high-level events, respectively. 

(2) In spite of the advantages they show in certain aspects, either GP models or HDP ones have inherent limitations. To overcome their respective drawbacks, the HDP models and the GP ones are used to constitute a new model in this thesis. At first, an unsupervised non-parametric generative HDP model is utilized to learn the typical activities and traffic states in a given surveillance video. Afterwards, a training dataset is constructed based on the learning results and then used to train the GP models for traffic state classification as well as the abnormal events detection. Furthermore, the temporal dependencies among traffic states are also integrated with the GP classifier to enhance the classification accuracy. 

(3) Two QMUL Junction datasets and one MIT dataset are used as benchmarks to evaluate the proposed approach. The experimental results demonstrate that the approach outperforms other popular approaches in both classification accuracy and computation efficiency. In particular, the improved GP classifier is capable to correct the falsely-classified clips by the original GP classifier. Besides, the abnormal events are roughly divided into three categories for a better detection: (i) the rarely occurring motions that do not belong to any typical activities; (ii) the conflicting activity that should not have occurred in a specific traffic state; and (iii) the state that has very low probability of transitioning from any previous state. Moreover, the detected abnormal events can be readily localized by taking advantage of the position information of visual words. 

However, there are some limitations in our current framework. Firstly, the proposed model merely takes into account the simple temporal dependencies within a clip in detecting conflicting activities. It could result in poor performances of abnormality detection in scenes, of which the traffic state is quite obscure because of the absence of traffic lights. Therefore, it is significant to temporally correlate activity with one another in a sophisticated manner. Secondly, the proposed framework does not include the detection and tracking. This could lead to undesired side effect when identifying subtle abnormal events among other common events. To address this problem, to incorporate the detection and tracking techniques into this proposed framework could provide a promising tool. The last but not the least, the long video sequence is arbitrarily divided into short clips in this thesis. As a consequence, the activities which occur in consecutive frames might be divided into two different clips or a clip might involve the state transition and thus leading to false abnormality detection. In the future, a more effective and non-intrusive method is of great interest to divide the long video sequence.
